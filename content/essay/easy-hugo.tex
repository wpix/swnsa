% Created 2020-04-15 Wed 03:14
% Intended LaTeX compiler: pdflatex
\documentclass[11pt]{article}
\usepackage[utf8]{inputenc}
\usepackage[T1]{fontenc}
\usepackage{graphicx}
\usepackage{grffile}
\usepackage{longtable}
\usepackage{wrapfig}
\usepackage{rotating}
\usepackage[normalem]{ulem}
\usepackage{amsmath}
\usepackage{textcomp}
\usepackage{amssymb}
\usepackage{capt-of}
\usepackage{hyperref}
\author{Congying Wang}
\date{\today}
\title{}
\hypersetup{
 pdfauthor={Congying Wang},
 pdftitle={},
 pdfkeywords={},
 pdfsubject={},
 pdfcreator={Emacs 26.3 (Org mode 9.2.6)}, 
 pdflang={English}}
\begin{document}

\tableofcontents \clearpage\sout{+}
title = "Writing Hugo pages in org-mode with Easy-Hugo"
bookmarks = ["emacs","org","hugo"]
date = "2020-04-08"
type = "essay"
\sout{+}

\section{Hugo \& org-mode}
\label{sec:org4936cea}

Both of my sites are written in Hugo as it is unbelievably fast in terms of building websites. It can deploy thousands of \textbf{.md} files, one per page, within a second, meaning I can almost write markdown documents and check the generated html page (and a complete version of the entire website) almost simultaneously. This is a critical feature to me since I constantly change blog entries and edit site parameters as a total newbie. 

A year ago, when I was using Sublime and BBedit, markdown formats worked just great, straight-forward and easy-to-remember syntax. With tree structured sidebar integrated in both editors, jumping between folders and files were effortlessly. A year after, though, I have dig into another rabbit hole 🐇 (Emacs) and way too deep. Getting used to org-mode also means being awkward with other text editors, and worse, with other file types. I find it a bit painful writing my last post with markdown, even with a powerful Emacs markdown-mode.

(A ton of guides on how to build a site with Hugo are available a google away. For this entry I won't touch on this subject.) The solution I found to integrating org-mode and Hugo is the \href{https://github.com/masasam/emacs-easy-hugo}{Emacs-easy-hugo} package. I noticed more discussions and posts on an alternative option with \href{https://github.com/kaushalmodi/ox-hugo}{ox-hugo} exporter but less on easy-hugo, so sharing my two cents about the latter in this entry. Incredibly easy to set up if you already have a functioning Hugo site running.

\section{Emacs-easy-hugo}
\label{sec:orgf2f3efe}
Several advantages of the easy-hugo package:

\begin{itemize}
\item easy to install via melpa
\item easy to configure
\item flexible and abundant org-mode features
\end{itemize}

I will start with the last advantage. For example, the literal programming integration allowed by org-babel. When I want to export a tree view for my blog folder to better demonstrate the configuration, there is no need to open terminal, execute the command, and paste the result back into org-mode. In stead, I can use the code block to export the result directly. 

\subsection{Org-babel}
\label{sec:org1e3687f}
The original code block used in org-mode is:

\begin{verbatim}
#+begin_src shell
cd ~/hugo/swnsa
tree -v --charset utf-8
#+end_src
\end{verbatim}

And the output automatically inserted in the org-mode file is shown as:
\begin{center}
\begin{tabular}{llllll}
├── & archetypes &  &  &  & \\
│   & └── & default.md &  &  & \\
├── & config.toml &  &  &  & \\
├── & content &  &  &  & \\
│   & ├── & about.md &  &  & \\
│   & ├── & albums &  &  & \\
│   & │   & └── & \_index.md &  & \\
│   & ├── & archives &  &  & \\
│   & │   & └── & \_index.md &  & \\
│   & ├── & bookmarks &  &  & \\
│   & │   & └── & \_index.md &  & \\
│   & ├── & essay &  &  & \\
│   & │   & ├── & \_index.md &  & \\
│   & ├── & photo &  &  & \\
│   & │   & ├── & \_index.md &  & \\
│   & └── & tags &  &  & \\
│   & └── & \_index.md &  &  & \\
├── & layouts &  &  &  & \\
│   & ├── & \_default &  &  & \\
│   & │   & ├── & index.xml &  & \\
│   & ├── & archives &  &  & \\
│   & │   & └── & archives.html &  & \\
│   & ├── & essay &  &  & \\
│   & │   & └── & single.html &  & \\
│   & ├── & index.html &  &  & \\
│   & ├── & page &  &  & \\
│   & │   & └── & single.html &  & \\
│   & ├── & partials &  &  & \\
│   & └── & taxonomy &  &  & \\
\end{tabular}
\end{center}
\ldots{}\ldots{}

I cannot elaborate enough how deeply intrigued I am by this brilliance. 

\subsection{Easy-hugo easy-configuration}
\label{sec:org5998df5}
Based on the above folder structure, below is the configuration added in my init file. In fact it is for two sites, which demonstrate another nice feature of easy-hugo, making managing and deploying multiple sites a breeze.

\begin{verbatim}
(setq easy-hugo-root "~/hugo/")
(setq easy-hugo-sshdomain "mini")

(setq easy-hugo-bloglist
	;; blog
      '(((easy-hugo-basedir . "~/hugo/swnsa/")
	 (easy-hugo-url . "http://ying-ish.com")
	 (easy-hugo-postdir "content/essay/"))
	;; blog2 for photos
	  ((easy-hugo-basedir . "~/hugo/mini/")
	   (easy-hugo-url . "https://mini.ying-ish.com")
	   (easy-hugo-postdir "content/photo/"))))

(setq easy-hugo-server-flags "-D")
\end{verbatim}

Single blog site is even easier to set up following the GitHub \href{https://github.com/masasam/emacs-easy-hugo\#sample-configuration}{readme}.

\section{Write and deploy}
\label{sec:org1964176}
To write in org-mode, simply activate easy-hugo mode, start a new post with \textbf{.org} extension. That's all you need to do. At any stage, easy-hugo can convert org format to markdown nicely. (It is not actually convert the format of the original org file, but more like a filter or an temporary exporter.)

This site is powered by \href{https://www.netlify.com/}{Netlify} with auto-deploy from the \href{https://github.com/wpix/swnsa}{GitHub repository} . I highly recommend this setting as, first, it is free, and second, Netlify is really fast. Without Netlify, I may need to wait for 2-3 minutes for GitHub page to deploy but with Netlify it never cost more than 30 seconds. 

I have a script to automate the git update which only cost 2 key strokes. But with Emacs-magit it is also super fast. 

Save, commit, and done.
\end{document}
